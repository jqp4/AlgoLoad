\begin{abstract}
% The abstract should summarize the contents of the paper
% using at least 70 and at most 150 words. It will be set in 9-point
% font size and be inset 1.0 cm from the right and left margins.
% There will be two blank lines before and after the Abstract. \dots
% We would like to encourage you to list your keywords within
% the abstract section using the \keywords{...} command.

This paper describes a new version of the AlgoView system for 3D visualization and interactive analysis of information graphs of algorithms. The developed system consists of two interacting parts: a functional computational framework and an environment for 3D visualization and interactive analysis of graph representations. The Algolang language is used as the input language of the system, which allows describing the fine information structure of a wide class of computational algorithms. The implemented visualization system has successfully proved itself in providing a detailed visual representation of the internal structure of algorithms, facilitating the process of analyzing their properties and parallelization possibilities. The AlgoView system eliminates the need for manual visualization of graphs and allows researchers to focus on analyzing the properties of the algorithms themselves. In addition, this system can be used in the educational process of higher education institutions to study the properties of algorithms, as well as for the preparation of scientific materials with high-quality examples of representations of the described information graphs.

\keywords{ Algorithm $\cdot$ 
           Information graph $\cdot$ 
           Parallel structure $\cdot$ 
           Parallel form $\cdot$ 
           Level parallel form $\cdot$ 
           AlgoView $\cdot$ 
           Visualization $\cdot$ 
           Web }

\end{abstract}
